\section{Conclusion}

This thesis investigates the Grueneisen effect, its application to photoacoustic microscopy and the comparison to optical resolution photoacoustic microscopy (OR-PAM). Furthermore, a setup to detect ultrasonic pressure waves by light is designed, built and characterized. \\
\\
Through a stepwise heating of OrangeG colored water and paprika powder colored rizinus oil, it is shown that the photoacoustic amplitude either rises (in case of water) or falls (in case of oil). This behavior is referred to the Grueneisen effect and is proven in figure \ref{fig:measuredGRproof}.\\
The dual pulse technique (section \ref{sec:dualPulseTechnique}) is introduced to make the Grueneiseneffect applicable to PAM. The images taken in section \ref{sec:ORGRcomp} show the additional possibilities of GR-PAM compared to OR-PAM, in particular the thermal contrast mechanism (figure \ref{fig:imgGRproof}) and the sectioning capabilities (figure \ref{fig:imgGRblackleaf}). Moreover, an improvement of 7~$\%$ in the lateral resolution, of GR-PAM in relation to OR-PAM, is determined. However, the theoretical improvement of $\sqrt{2}$ could not be reached.\\
Besides the travel time of the ultrasonic wave is used to gain structural information of different materials in OR-PAM (section \ref{sec:topoPAI}). Thereby the data of the black leaf sample is used to determine the sample tilt compared to the scan plane, shown in figure \ref{fig:sampletilt}.\\
In the second part a system, based on a Fabry–P\'{e}rot interferometer, is developed to detect ultrasonic pressure waves by light (section \ref{sec:OUSD}). The task was to built a compact and movable ultrasonic detector. This is realized by replacing the free jet implementation, of the excitation and detection laser supply, through optical fiber and reducing the size of the mirrors and housing (section \ref{sec:sysDev}). The functionality is tested and proved in section~\ref{sec:OUSDresults}. The built setup has a finesse of about 6 to 7 and a estimated cut off frequency of 15.04~$MHz$.
Additionally the maximum pressure amplitude of several ultrasonic piezo transducers is determined by a needle hydrophone and a Mach-Zehnder interferometer setup (table \ref{tab:pressureVal}). Overall, all requirements given for the work are met.\\ \\

In order to improve the scan speed of the OR-PAM setup, described in section \ref{sec:ORPAMsetup}, the positioning stages could be replaced by a scanning galvo mirror system. \\
During the operation of the OUSD the problem emerged that small air bubbles are accumulate at the center of the concave mirrors. This could be solved by a piezoelectric element in the $kHz$-range, which is coupled onto the system to degas the media between the mirrors. \\
A further improvement would be the use of dielectric-coated plano-concave mirrors for the Fabry–P\'{e}rot interferometer. This leads to an increased finesse and results in a higher sensitivity of the OUSD. \\
In order to work with mirror distances closer to the  concentric case (see figure \ref{fig:resStability} and \ref{fig:resLength}), a refine of the mirror mounting system would lead to a more preciser alignment.\\
This thesis shows a prove of concept for a compact OUSD system, though further characterization measurements towards the transfer function should be done to give a more precise value for the cutoff frequency.  \\ 
\\
 In summery, GR-PAM is an useful addition to common photoacoustical methods. It can provide additional information about a sample and increase the contrast, if substances with differing $G$ are included.\\
 Furthermore, the optical detection of ultrasonic pressure waves can contribute a good alternative to common piezoelectric transducers.