\section{Introduction}

In todays medical diagnosis a lot of different imaging technologies are available to determine and examine various types of diseases. Whether it is magnetic resonance tomography (MRT), to display organs and tissue, or optical microscopy to figure out bacteria. At some point, for example in conventional microscopy, it is a necessity to use contrast agents to make things visible or in case of MRT, it is to expansive to gain benefit from it \cite{grun:integrating}. In order to meet this facts and overcome drawbacks the field of photoacoustic imaging is an very uprising field in medical diagnostics. The detected value is not back scattered or transmitted light, but a pressure wave generated by the sample itself through the interaction with light. This makes it a method with a lot of possibilities, too. Furthermore, it is non-invasive because the needed light intensity is below the damage threshold of biological materials. \\
Another advantage is the use of natural contrast agents such as melanin or hemoglobin. Due to the contrast mechanism is founded in the difference of the optical absorption~coefficient~$\mu_a$. Based on this effect two main fields have emerged, the photoacoustic microscopy (PAM) and the photoacoustic tomography (PAT) \cite{YAO201487}. \\
\\
This thesis focuses on two topics in the field of photoacoustic microscopy. At the beginning the photoacoustic effect and its technical realization in optical resolution photoacoustic microscopy (OR-PAM) gets described as state of the art technique. \\
Afterwards the Grueneisen effect and its application to microscopy, called Grueneisen relaxation photoacoustic microscopy (GR-PAM), gets introduced. Including a prove of the effect and an analysis of GR-PAM versus OR-PAM in view of additional contrast capabilities and sectioning qualities. The measurements are performed ex-vivo.\\
The second part of the thesis presents the construction and characterization of an optical ultrasonic transducer system. The described system is a further development of the one described by Gratt \cite{GrattSibylle:Dis} with focus on building a compact and movable system.\\

